
Maximum likelihood estimation can lead to severe over-fitting if complex models (e.g. polynomial regression models of high order) are fit to datasets of limited size.
Common approachs to prevent over-fitting is to add a regularization term to the error function or to perform corss-validation.

\begin{framed}
Bayesian treatment of linear regression avoid over-fitting problem of maximum likelihood, and which will also leads to automatic methods of determining model complexity using the training data alone.
\end{framed}

For a Bayesian treatment of linear regression we need a prior probability distribution over model parameters $w$.

\begin{equation}
 p(\vm{w}) = \N(\vm{w}|\vm{m}_0, \vm{S}_0)
\end{equation}

Due to the choice of a conjugate Gaussian prior, the posterior will also be Gaussian.
We can evaluate this distribution by the usal procedure of completing the square in the exponential, and then finding the normalization coefficient using the standard result for normalized Gaussian.
The work for deriving the general result is at equation (\ref{eq:post})

\begin{equation}
 p(\vm{w}|\vm{t}) = \N(\vm{w}|\vm{m}_N, \vm{S}_N)
\end{equation}

where 

\begin{equation}
 \vm{m}_N = \vm{S}_N (\vm{S}_0^{-1} \vm{m}_0 + \beta \vm{\Phi}^T \vm{t})
\end{equation}

\begin{equation}
 \vm{S}_N^{-1} = \vm{S}_0^{-1} + \beta \vm{\Phi}^T\vm{\Phi}
\end{equation}

For simplicity, we consider a zero-mean isotropic Gaussian prior governed by a single precision paramater $\alpha$ so that

\begin{equation}
 p(\vm{w}) = N(\vm{w}|\vm{0}, \alpha^{-1} \vm{I})
\end{equation}

then corresponding posterior distribution over $\vm{w}$ is then 


\begin{equation}
 \vm{m}_N = \beta  \vm{S}_N\vm{\Phi}^T \vm{t}
\end{equation}

\begin{equation}
 \vm{S}_N^{-1} = \alpha \vm{I} + \beta \vm{\Phi}^T\vm{\Phi}
\end{equation}


To obtain an analytical solution we will treat $\beta$ as a known constant.
Note that in supervised learning problems such as regression we are not seeking to model the distribution of the input variables, so we will treat the input $\vm{x}$ as a known constant.


\begin{figure}[H]
\begin{subfigure}[t]{0.32\textwidth}
\includegraphics[width=\textwidth]{../figures/pdf/linearModel.pdf}
\end{subfigure}
\end{figure}


\begin{algorithm}[H]  
  \caption{Posterior for Linear Regression Model}
  \label{alg:posterior_for_linear_regression_model}
  %\lstinputlisting{../ablr/linear/posterior.py}
\begin{lstlisting}
def posterior(alpha, beta, t, Phi):
    S_N_inv = alpha * np.eye(Phi.shape[1]) + beta * Phi.T.dot(Phi)
    S_N = np.linalg.inv(S_N_inv)
    m_N = beta * S_N.dot(Phi.T).dot(t)
    return m_N, S_N
\end{lstlisting}
\end{algorithm}

\begin{figure}[H]

\begin{subfigure}[t]{0.32\textwidth} 
\caption*{Likelihood} 
\end{subfigure}
\begin{subfigure}[t]{0.32\textwidth}
\caption*{Prior/Posterior} 
\includegraphics[width=\textwidth]{../figures/pdf/example1_posterior_0.pdf} 
\end{subfigure}
\begin{subfigure}[t]{0.32\textwidth}
\caption*{Data space} 
\includegraphics[width=\textwidth]{../figures/pdf/example1_dataSpace_0.pdf} 
\end{subfigure}


\begin{subfigure}[c]{0.32\textwidth}
\includegraphics[width=\textwidth]{../figures/pdf/example1_likelihood_1.pdf} 
\end{subfigure}
\begin{subfigure}[c]{0.32\textwidth}
\includegraphics[width=\textwidth]{../figures/pdf/example1_posterior_1.pdf} 
\end{subfigure}
\begin{subfigure}[c]{0.32\textwidth}
\includegraphics[width=\textwidth]{../figures/pdf/example1_dataSpace_1.pdf} 
\end{subfigure}

\begin{subfigure}[c]{0.32\textwidth}
\includegraphics[width=\textwidth]{../figures/pdf/example1_likelihood_2.pdf} 
\end{subfigure}
\begin{subfigure}[c]{0.32\textwidth}
\includegraphics[width=\textwidth]{../figures/pdf/example1_posterior_2.pdf} 
\end{subfigure}
\begin{subfigure}[c]{0.32\textwidth}
\includegraphics[width=\textwidth]{../figures/pdf/example1_dataSpace_2.pdf} 
\end{subfigure}

\begin{subfigure}[c]{0.32\textwidth}
\includegraphics[width=\textwidth]{../figures/pdf/example1_likelihood_3.pdf} 
\end{subfigure}
\begin{subfigure}[c]{0.32\textwidth}
\includegraphics[width=\textwidth]{../figures/pdf/example1_posterior_3.pdf} 
\end{subfigure}
\begin{subfigure}[c]{0.32\textwidth}
\includegraphics[width=\textwidth]{../figures/pdf/example1_dataSpace_3.pdf} 
\end{subfigure}

\begin{subfigure}[c]{0.32\textwidth}
\includegraphics[width=\textwidth]{../figures/pdf/example1_likelihood_4.pdf} 
\end{subfigure}
\begin{subfigure}[c]{0.32\textwidth}
\includegraphics[width=\textwidth]{../figures/pdf/example1_posterior_4.pdf} 
\end{subfigure}
\begin{subfigure}[c]{0.32\textwidth}
\includegraphics[width=\textwidth]{../figures/pdf/example1_dataSpace_4.pdf} 
\end{subfigure}
\caption{Ilustration of a sequential Bayesian learning for a simple linear model of the form $y(x,\vm{w})= w_0 + w_1 x$}
\end{figure}










